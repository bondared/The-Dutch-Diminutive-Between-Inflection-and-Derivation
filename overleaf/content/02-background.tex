\chapter{The Nature of the Dutch Diminutive}\label{background}
"Diminutive" is a term that is intuitively understandable as having something to do with size. However, in order to lay the groundwork for this thesis, we need to establish the parameters typical for diminutive wordforms across languages and use those to arrive at a prototypical diminutive wordform. We will start with a working definition adapted from \cite{Schneider+2003} and use it to examine the formal and semantic properties associated with diminutives. 

\section{The Diminutive}

\textbf{Start with the general meaning; broadly define the group}

At the start of his seminal \citeyear{Schneider+2003} work on English diminutives, \citeauthor{Schneider+2003} provides the traditional applied definition of diminutives as "words which denote smallness and possibly also expressing an attitude" \parencite[p.4]{Schneider+2003} and outlines the formal and semantic criteria used to describe a diminutive form. 

\subsection{Formal Criteria}
\textbf{Focus on the formal  component, look at such criteria as input category, output category, morphological process}

\subsection{Semantic Criteria}
\textbf{Segue to the semantic component, link the obligatory narrow-semantic denotation to the structural exponents, discuss the broad-semantic/pragmatic attitudinal aspect.}

\subsection{Diminutives in Germanic Languages}
\begin{itemize}
\item Broad diminutive formation patterns in some Germanic Languages (\cite{Alexiadou+Lohndal}, \cite{Schneider+2003})
\item Diminutives in Dutch as compared to other Germanic Languages (\cite{Booij+2000}, \cite{DeBelder+2022}) 
\end{itemize}


Starting with the semantic criteria, While the obligatory semantic component typically expresses smallness, the optional attitudinal part can vary between positive and negative attitude expression. Consider the example in (\ref{ex:kereltje}) using the Dutch diminutive suffix \textit{-tje}: depending on the context, \textit{kerel-tje} "little lad" can be used either as a term of endearment or an expression of dismissiveness.

The same holds true of \textit{little lad} in the English gloss. Here, \textit{little} seems to primarily express an attitudinal component, and using \textit{small} to express a lexical meaning similar to that of \textit{little} does not yield the same interpretation, as shown in (\ref{ex:littlelad}). This function of \textit{little} as a tool of diminutive formation in English is reflective of a common trend across languages.

\begin{exe}
\ex \label{ex:littlelad}
\begin{xlist}
\ex He is a strange little lad (dismissively)
\ex \#He is a strange small lad (dismissively)
\end{xlist}
\end{exe}

\begin{exe}
\ex \label{ex:kereltje} 
\begin{xlist}
\ex \gll
Hij is een raar kereltje.\\
He	is	a  strange	lad-DIM\\
\trans "He is a strange little lad" (dismissively)

\ex \gll
Hij is een lief kereltje.\\
He	is	a  sweet	lad-DIM\\
\trans "He is a sweet little lad" (endearingly)
\end{xlist}
\end{exe}

While part of the meaning of diminutive forms like \textit{kereltje} seems to be conditioned by their immediate context, there is no clear-cut split between a purely semantic scalar meaning component and a purely pragmatic attitudinal one. For example, the German word \textit{Väterchen}, formed from \textit{Vater} "father" with a diminutive suffix -\textit{chen}, acts as a term of endearment regardless of context. The same applies to the word \textit{Mütterlein}, similarly formed from \textit{Mutter} "mother" via the suffix -\textit{lein}: both mean something like "dear father" and "dear mother", respectively, without any need for any further modification. In addition, such forms seem to lack the scalar component [+small] in their primary interpretation \parencite{Schneider+2003}. This leads us to two conclusions about the scalar/attitudinal split:

Based on the previous example, it quickly becomes clear that a diminutive's lexical meaning is not always purely compositional, i.e. a diminutive wordform doesn't always denote a smaller version of some entity. Indeed, sometimes the resulting meaning is so opaque that it has nothing to do with expressing a difference in size anymore. Consider the Italian diminutive suffix \textit{-in-} in (\ref{ex:casino}): its impact on the meaning of the underlying root seems to range from purely compositional in (\ref{ex:casino1}) to semi-opaque in (\ref{ex:casino2}) and (\ref{ex:casino3})\footnote{Both a cell phone and a bread roll are arguably smaller, more compact versions of a telephone and a bread loaf, respectively; however, both \textit{telefonino} and \textit{panino} denote objects with other characteristics beyond their relative size, making their meaning partially non-compositional.} to purely non-compositional in (\ref{ex:casino4}). This pattern is also attested in such languages as Dutch, Spanish, Polish, Modern Hebrew, and Tunisian Arabic \parencite[see][for detailed examples]{DeBelder+etal+2014}.

\begin{exe}
\ex \label{ex:casino} 
\begin{xlist}
\ex \label{ex:casino1} \gll
gattino \\
cat-DIM \\
\trans "small cat", "kitten"

\ex \label{ex:casino2} \gll
telefonino \\
phone-DIM \\
\trans "cell phone"

\ex \label{ex:casino3} \gll
panino \\
bread-DIM \\
\trans "bread roll, sandwich"

\ex \label{ex:casino4} \gll
casino \\
house-DIM \\
\trans "brothel, hunting lodge"
\footnote{Note that there exists a purely compositional noun \textit{casina} "small house" that even employs the same suffix \textit{-in-}. The difference here is the induced change in noun gender that seems to correlate with a change in meaning.}
\end{xlist}
\end{exe}

As quickly outlined above, diminutive semantics can be surprisingly complex

Formally, then, a "prototypical" diminutive is a nominal wordform derived from a noun by suffixation; semantically, the base noun's meaning is modified by the suffix to include the feature [+small] and optionally either [+positive] or [+negative] for the attitudinal component. Given the previously described non-nominal formations, a prototypical diminutive morpheme is a derivational suffix meaning "small" that attaches to a word and possibly, but not obligatorily, nominalises it.

\section{The Dutch Diminutive Suffix}
\label{sec:dutchdimsuffix}
\subsection{Peculiarities}
\textbf{Focus on the Dutch suffixal diminutive (\cite{VanderHulst+2008}, \cite{DeBelder+2022}), discuss the different semantic interpretations}
\subsection{Inflection or Derivation?}
\textbf{Outline the inflection/derivation split, showcase how the Dutch diminutive suffix fulfills some criteria on each side of the divide (\cite{Schneider+2003}, \cite{Booij+2000})}