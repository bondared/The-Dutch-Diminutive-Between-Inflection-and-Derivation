\chapter{Diminutives around the world}\label{background} 
In order to lay the groundwork for this thesis, we need to establish the parameters typical for diminutive wordforms across languages and use those to arrive at a prototypical diminutive wordform. We will start with a working definition adapted from \cite{Schneider+2003} and use it to examine the formal and semantic properties associated with diminutives. 

\section{The Diminutive}

At the start of his seminal \citeyear{Schneider+2003} work on English diminutives, \citeauthor{Schneider+2003} provides the traditional applied definition of diminutives as "words which denote smallness and possibly also expressing an attitude" \parencite[p.4]{Schneider+2003} and outlines the criteria used to describe a diminutive form, distinguishing between the formal and the semantic ones. 

\subsection{Formal Criteria}
\label{subsec:2-formal-crit}
The formal criteria describe the variables involved in the process of diminutive formation, broadly conceptualised as a function that takes a word and performs an operation on it in order to produce a diminutive formally and/or semantically related to the initial word. The variables involved in this function, then, are the input and output word, as well as the type of unilateral morphological process that transforms one into the other. Of particular theoretical interest are the possible changes in word class between the input and the output word induced by the morphological process. 

There is considerable cross-linguistic variation in terms of what can function as an input to the diminutive function. Nouns are intuitively the go-to category when one thinks of diminutive formation: words such as the English \textit{dog}-\textit{doggie} "dog+DIM", or the German \textit{Katze}-\textit{Kätzchen} "cat+DIM" are ubiquitous across languages. Additional possible open-class categories include adjectives (cf. Russian \textit{старенький} "old+DIM", from \textit{старый} "old", or Italian \textit{azzurrino} "blue+DIM, pale blue", from \textit{azzurro} "sky blue"), adverbs (cf. Dutch \textit{netjes} "neatly+DIM", from \textit{net} "neatly"), and even verbs (cf. German \textit{drängeln} "push+DIM, jostle", from \textit{drängen} "press, push", and Italian \textit{fisciettare} "whistle+DIM", from \textit{fischiare} "whistle"). Closed-class categories such as interjections and prepositions are also known to sporadically provide the base of a diminutive form: cf. Russian \textit{приветик} and German \textit{hallöchen} "hello+DIM", respectively from \textit{привет} and \textit{hallo} "hello", as well as Dutch \textit{uitje} "outing" from \textit{uit} "out" and \textit{ommetje} "short walk (around the block)" from \textit{om} "around". In even rarer cases, whole phrasal structures can be co-opted by the process of diminutive formation, such as in the case of the Dutch \textit{onderonsje} "private conversation", from \textit{onder} "under, among" + \textit{ons} "us". Despite this broad selection of available categories, the noun forms the most frequent one by far and considered to be the prototypical input word class by \cite{Schneider+2003}, who establishes the following hierarchy:

\begin{exe}
\ex \label{ex:inputhierarchy}
Word class input in diminutive formation (adopted from \citeauthor{Schneider+2003} \citeyear{Schneider+2003}:6)
\begin{enumerate}
\item[i.] open-class words > closed-class words
\item[ii.] nouns and adjectives > verbs and adverbs
\item[iii.] nouns > adjectives
\end{enumerate}
\end{exe}

If the prototypical base of a diminutive word is a noun, then what is the prototypical output? Most of the examples so far have exhibited no change in category between input and output. However, there is some evidence of nominalisation such as the Dutch \textit{uitje} and \textit{ommetje}, where prepositional bases are changed into nouns. The same process often applies to deadjectival forms: cf. English \textit{sweet}-\textit{sweetie}, Dutch \textit{groen} "green" - \textit{groentje} "novice, rookie", or Italian \textit{rosso} "red, ginger" - \textit{rossino} "redhead". Based on these observations, it seems that word class is either retained from the base form or shifts to nominal. \citeauthor{Schneider+2003} (\citeyear{Schneider+2003}) observes that the former tends to apply as a general rule, with the latter otherwise being the case. A typical diminutive form is therefore a noun produced from a noun. However, word class changes are common as well, resulting in nominalised forms.

With the input and output parameters having been examined, we can focus on the selection of morphological processes that are responsible for the process of diminutive formation across languages. Again, based on the examples presented so far, it would be safe to assume that affixation (or, more precisely, suffixation) is the sole engine behind this process. However, as the non-exhaustive sample in (\ref{ex:dimformpatterns}) shows, this is far from the whole picture even in the case of a single language like English:

\begin{exe}
\ex \label{ex:dimformpatterns}
A sample of English diminutive formation mechanisms 
\begin{xlist}
    \ex suffixation: \textit{dog} - \textit{doggie}
    \ex prefixation: \textit{pig} - \textit{minipig}
    \ex truncation: \textit{Elizabeth} - \textit{Liz}
    \ex (rhyming) reduplication: \textit{Jeff Banks} - \textit{Jeff Banksy-Wanksy}
    \ex compounding: \textit{boy} - \textit{baby boy}
    \ex analytic formation: \textit{lady} - \textit{little lady}
    \ex any combination thereof\footnote{For truly outstanding examples of this in action, the reader is kindly referred to the \href{https://abitoffryandlaurie.co.uk/sketches/pooch}{transcript of the "Pooch" sketch} featured on the UK TV show \textit{A bit of Fry and Laurie}.}
\end{xlist}
\end{exe}

The formation mechanisms in (\ref{ex:dimformpatterns}) fall into two overarching categories: those that directly modify the root or stem of the base word, and those that combine another word like \textit{little} with the stem to seemingly form a whole phrase. The function of \textit{little} and similar terms as a tool of diminutive formation in English is reflective of a common trend across languages. The categorical split is observed in \citeauthor{Schneider+2003} (\citeyear{Schneider+2003}), who uses the term "synthetic diminutive formation" for the former and "analytic diminutive formation" for the latter broad category. 

\citeauthor{Schneider+2003} makes an additional distinction between derivational and inflectional affixation: while purely derivational affixes do not effect a change in grammatical information such as noun class or gender, those that do are considered to belong in the domain of inflection. While English lacks gender, its sister language German does not, and can provide a good example. All German nouns have an associated gender value, such as \textit{Löffel} "spoon" being coded masculine and \textit{Gabel} "fork" feminine. However, in the process of diminutive formation involving either of the suffixes \textit{-chen} and \textit{-lein}, a mandatory gender switch takes place: \textit{Löffellein} "spoon+DIM" and \textit{Gäbelchen} "fork+DIM" are both neuter, constituting a case of inflectional suffixation. \citeauthor{Schneider+2003} goes on to claim in passing that the distinction between inflectional and derivational affixation routes in diminutive formation applies parametrically for every language, implying that diminutive affixes are all either exclusively derivational or inflectional in any given language, but notes that this claim requires more evidence, preferably from non-Indo-European languages.

Both distinctions will be touched upon again in the subsequent sections, with the inflectional-derivational distinction especially discussed in Subsection \ref{subsec:2-infl-deriv} and Section \ref{sec:3-split-dim}. For now, it suffices to establish derivational suffixation as the most frequent mechanism of diminutive formation across languages (\citeauthor{Schneider+2003} \citeyear{Schneider+2003}). Formally, then, a prototypical diminutive is a noun formed from a noun by derivational suffixation. With the formal parameters now outlined and their default values established, let us consider the patterns of meaning typically associated with diminutives next.

\subsection{Semantic Criteria}
\label{subsec:2-semantic-crit}
The linguistic term "diminutive" intuitively has something to do with size comparison, and while precise definitions of the whole category may vary depending on the theoretical framework, the scalar aspect of diminutive meaning is invariably the central one. This is captured in the traditional applied definition provided by \citeauthor{Schneider+2003}: diminutives have long been thought of as "words which denote smallness and possibly also expressing an attitude" (\citeauthor{Schneider+2003} \citeyear{Schneider+2003}:4), with the attitudinal component expressing either a positive or a negative value. From this definition alone, one can visualise each of the two components as expressing a scalar relationship: a \textit{doggie} is a small \textit{dog}, while \textit{Lizzie} is a term of endearment for someone called \textit{Elizabeth}, indicating a positive attitude of the speaker towards them. The notion of scalarity being central to diminutive meaning warrants a deeper examination below.

As far as the size component is concerned, the semantic relationship between a diminutive and its base word is straightforwardly that of hyponymy: a \textit{doggie}, being a small \textit{dog}, is part of the category \textsc{dog}. Therefore, diminutives tend to express relative smallness as defined by the semantic category they belong to. For animate categories, this fact carries an additional consequence of diminutives typically denoting non-adult members (cf. \textit{duck}-\textit{duckling}, \textit{prince}-\textit{princeling}), with the latter example additionally implying immaturity or irrelevance. In purely semantic terms, the diminutive denotes something in some way inferior by comparison, implying a stable relationship between its denotation and the category member it is being compared to.

However, this observation does not mean that a speaker invariably refers to a small duck when saying the word \textit{duckling}, or even perceives the entity they are referring to as small: according to \citeauthor{Schneider+2003} (\citeyear{Schneider+2003}), the speaker's intent plays a crucial role in the total meaning of a diminutive form. In other words, the choice to refer to something as small or inferior is sometimes more relevant than its actual relative size. The example of \textit{princeling} above serves as a good showcase of this, usually being meant derogatively. Therefore, it seems that even the most prototypical aspect of diminutive meaning is somewhat modulated by pragmatic considerations.

The attitude component, established in the above definition as optional, describes the speaker's feelings about the referent. This part of diminutive meaning, too, is inherently scalar: the referent is regarded either more or less favourably by the speaker than some default value. It stands to reason, then, that this component would also be subject to pragmatic effects. Indeed, this is often the case to an even greater degree, with the amount of (dis-)favour sometimes being entirely dependent on the context. Consider the example in (\ref{ex:kereltje}) using the Dutch diminutive suffix \textit{-tje}: \textit{kereltje} "lad+DIM" can be used either as a term of endearment or an expression of dismissiveness.

\begin{exe}
\ex \label{ex:kereltje} 
\begin{xlist}
\ex \gll
Hij is een raar kereltje.\\
He	is	a  strange	lad-DIM\\
\trans "He is a strange little lad" (dismissively)

\ex \gll
Hij is een lief kereltje.\\
He	is	a  sweet	lad-DIM\\
\trans "He is a sweet little lad" (endearingly)
\end{xlist}
\end{exe}

The same holds true of \textit{little lad} in the English gloss. Here, \textit{little} seems to primarily express an attitudinal component, and using the purely size-denoting \textit{small} to express a lexical meaning similar to that of \textit{little} does not yield the same interpretation, as shown in (\ref{ex:littlelad}):

\begin{exe}
\ex \label{ex:littlelad}
\begin{xlist}
\ex He is a strange little lad. (dismissively)
\ex *He is a strange small lad. (dismissively)
\end{xlist}
\end{exe}

Despite the previous examples, attitude is not purely relegated to pragmatics, and some diminutive forms even code it as their primary semantic component. For example, the German word \textit{Väterchen} "father+DIM", from \textit{Vater} "father", acts as a term of endearment regardless of context. The same applies to the word \textit{Mütterlein} "mother+DIM", similarly formed from \textit{Mutter} "mother" via the suffix \textit{-lein}: both mean something like "dear father" and "dear mother", respectively. If anything, "dear X" is the only available meaning for either of the two, as glossing them with "small X" is unlicensed  (\citeauthor{Schneider+2003} \citeyear{Schneider+2003}).

The observations above paint a complex picture of the two primary diminutive meaning components: while both are inherently scalar, one seems to be primarily semantic in nature and the other is mostly governed by pragmatics. However, neither of the two allows for a clear-cut categorisation and often interplays with the other, leaving the prototypical meaning of a diminutive harder to pinpoint. \citeauthor{Schneider+2003} (\citeyear{Schneider+2003}) maintains that smallness is the prototypical component of diminutive meaning but simultaneously concedes that diminutive forms usually express something at "the interface between concepts of quantification and qualification, [combining] aspects of size and attitude" (\citeauthor{Schneider+2003} \citeyear{Schneider+2003}: 4).

An additional aspect of diminutive meaning that logically follows from their quantificational nature is their ability to induce shifts from mass to count nouns in some languages. \citeauthor{Wiltschko+2006} (\citeyear{Wiltschko+2006}) presents evidence from Dutch and German and argues that both languages are able to transform mass nouns like \textit{brood} / \textit{Brot} "bread" or even \textit{slaap} / \textit{Schlaf} "sleep" into their countable counterparts through diminutive formation. Consider the examples from German in (\ref{ex:mass-to-count-to-kind-to-item}): the word \textit{Brot} is a mass noun that acquires a kind reading whenever it is quantified by a process such as pluralisation, but this is as far as the shift goes. In order to talk about an individual roll of bread, the speaker has to form the diminutive \textit{Brötchen}. By contrast, other mass nouns like \textit{Schlaf} do not even have a kind reading, and depend fully on the diminutive for countability.

\begin{exe}
\ex \label{ex:mass-to-count-to-kind-to-item} 
Diminutive quantification in German (adopted from \citeauthor{Wiltschko+2006} \citeyear{Wiltschko+2006})
\begin{multicols}{2}
\begin{xlist}
\ex \gll
viel Brot\\
much bread\\
\trans "much bread"
\ex \gll
viele Brote\\
many-PL bread-PL\\
\trans "many kinds of bread"
\ex \gll
viele Brötchen\\
many-PL bread-DIM.PL\\
\trans "many bread rolls/sandwich"
\columnbreak
\ex \gll
viel Schlaf\\
much sleep\\
\trans "much sleep"
\ex \gll
*viele Schläfe\\
many-PL sleep-PL\\
\trans ~~~~~~
\ex \gll
viele Schläfchen\\
many-PL sleep-DIM.PL\\
\trans "many sleeps/naps"
\end{xlist}
\end{multicols}
\end{exe}

\citeauthor{Wiltschko+2006} thus ascribes an individuating function to the German and Dutch diminutive suffixes and notes that this effect is observed in other languages such as Yiddish, Ojibwa, Ewe, Baule, and Cantonese. She then compares the German diminutive suffix to numeral classifiers such as \textit{Stück} "piece" in \textit{12 Stück Brot} "12 pieces of bread" and uses the evidence of the shared pattern to argue that the suffix formally functions as a classifier as well. This has implications for the morphological structure of such diminutives, with the diminutive suffix proposed to act as the head of its phrase.

All the patterns of diminutive meaning discussed so far have been compositional in their nature: no matter the morphological process or degree of modification to the original meaning, it was still fairly transparently available. However, sometimes a formally diminutive word does not seem to have an independently-occurring base, e.g. German \textit{Kaninchen} "rabbit" or Dutch \textit{doetje} "softie". Words like this are called \textit{diminutiva tantum} and exhibit a wide range of meanings, rarely retaining the typical diminutive meaning components. In other cases the base word does exist, yet there is no observable relation between the denotations of the base and its diminutive anymore, e.g. Italian \textit{casino} "brothel", from \textit{casa} "house". A further complication is that words seemingly formed with the same suffix can span the range of fully transparent, as in (\ref{ex:casino1}) to fully opaque in (\ref{ex:casino4}).\footnote{Both a cell phone and a bread roll are arguably smaller, more compact versions of a telephone and a bread loaf, respectively; however, both \textit{telefonino} and \textit{panino} denote objects with other characteristics beyond their relative size, making their meaning partially non-compositional.} This pattern is attested in a wide variety of languages, from Dutch and Spanish to Polish and Tunisian Arabic (see \citeauthor{DeBelder+etal+2014} \citeyear{DeBelder+etal+2014} for further examples).
%\footnote{Note that there exists a purely compositional noun \textit{casina} "small house" that even employs the same suffix \textit{-in-}. The difference here is the induced change in noun gender; as discussed above, this would indicate that the suffix \textit{-in-} is an inflectional one. However, this quickly calls into question the parametric affixation claim from \citeauthor{Schneider+2003} (\citeyear{Schneider+2003}): either all Italian diminutive suffixes are }. The Italian diminutive suffix \textit{-in-} 
\begin{exe}
\ex \label{ex:casino} 
\begin{multicols}{2}
\begin{xlist}
\ex \label{ex:casino1} \gll
gattino \\
cat-DIM \\
\trans "small cat", "kitten"
\ex \label{ex:casino2} \gll
telefonino \\
phone-DIM \\
\trans "cell phone"
\columnbreak
\ex \label{ex:casino3} \gll
panino \\
bread-DIM \\
\trans "bread roll, sandwich"
\ex \label{ex:casino4} \gll
casino \\
house-DIM \\
\trans "brothel"
\end{xlist}
\end{multicols}
\end{exe}

The diminutive category, therefore, constitutes a broad semantic range, and the meaning of a diminutive is not always clearly discernible from its formal characteristics or the meanings of its constituent parts alone. Additionally, pragmatic considerations come into play even for some of the most straightforward cases of diminution. However, within the domain of semantic compositionality, a clear pattern of scalarity emerges, with the size and attitude components sharing the dominant position (\citeauthor{Schneider+2003} \citeyear{Schneider+2003}) among other possible aspects of meaning.

\subsection{Between Inflection and Derivation}
\label{subsec:2-infl-deriv}
One crucial categorical distinction that has been touched upon during the discussion of formal criteria in Subsection \ref{subsec:2-formal-crit} is that between inflectional and derivational diminutives. Recall that within the domain of synthetic diminutive formation, \citeauthor{Schneider+2003} (\citeyear{Schneider+2003}) sets derivational affixation as the prototypical morphological process and goes to suggest that inflectional and derivational affixation are mutually exclusive: in any chosen language, only one of the two processes is responsible for all of diminutive affixation. While the precise strength of the claim is debatable, as pointed out by \citeauthor{Schneider+2003} himself, the claim points to an important question: what makes a diminutive inflectional or derivational?

Traditionally, derivation and inflection have been thought of as distinct morphological processes, each one responsible for a particular linguistic function. According to \citeauthor{Booij+2000} (\citeyear{Booij+2000}), this functional difference is the crucial one for distinguishing between the two: derivation is employed to form new lexemes, i.e. word entries in the mental lexicon, while inflection is said to take existing lexemes and merely modify their form. Already an assumption becomes clear: a lexicon, i.e. a list of morphological primitives such as free and bound morphemes and/ possibly even morphologically complex forms with a set lexical meaning, is assumed to form a distinct part of the linguistic system. Derivation, then, is a process that operates on the lexicon and internally modifies it, producing new lexical entries; inflection takes lexical entries and modifies them, but does not expand the lexicon.

The fact that both processes access the lexicon, together with them often sharing the same formal means of expression (\citeauthor{Booij+2000} \citeyear{Booij+2000}), somewhat blurs the line between the two; yet they differ in regards to their typical behaviour along a few parameters. While listing and discussing all of those is beyond the purview of this thesis, the reader is referred to \citeauthor{Booij+2000} (\citeyear{Booij+2000}) for a concise overview and \citeauthor{Scalise+1986} (\citeyear{Scalise+1986}) for an in-depth analysis. What follows is a closer look at the parameters most relevant to diminutive formation.

 As previously observed, the process of diminutive formation can effect a change in word class between the base word and its diminutive form. Word class change also happens to be one of the major parameters between derivation and inflection, with inflectional processes typically seen as operating within syntactic categories while derivational ones optionally induce the change. Note that there are counterexamples when it comes to inflection, e.g. verb forms inflected for infinitive functioning as nouns, as well as participles being used as attributive adjectives (\citeauthor{Booij+2000} \citeyear{Booij+2000}). Nevertheless, the general pattern is that word class change is optional in cases of derivation and nearly unattested in cases of inflection.

 More extreme changes in meaning have also been traditionally associated with the domain of derivation; this is closely related to the notion of semantic transparency, with inflected forms regarded as comparatively more transparent (\citeauthor{Booij+2000} \citeyear{Booij+2000}). Subsection \ref{subsec:2-semantic-crit} has established that diminutives as a category can show a broad range of meanings, from transparent to opaque; nevertheless, the prototypical meaning of "small X" is reasonably transparent. With regard to the individuating capabilities of some members, sketched out in \citeauthor{Wiltschko+2006} (\citeyear{Wiltschko+2006}), a derivational analysis applies: \citeauthor{Scalise+1986} (\citeyear{Scalise+1986}) observes that changes in countability, as well as other semantic features such as animacy, are induced by processes of derivation.

The capability for recursivity and the relative position in the order of morphemes have also been used as criteria for demarcation. Derivational processes are said to have greater recursive capabilities. This follows from the observation that derivation is usually concatenative, with mutiple suffixes possibly forming new words iteratively (such as in \textit{dis-establish-ment-ari-an-ism}), while inflectional morphemes are often fusional, expressing multiple aspects of meaning in one surface form (e.g. \textit{-s} in \textit{buys} coding tense, person, and number all at once). In addition, inflection has been robustly observed to occur after derivation in processes of word formation, with the reverse ordering usually being clearly ungrammatical:      
This observation  while derivation iteratively performs its operations on stems, inflection takes the final stem and applies the grammatical meaning components to it, resulting in a fully-fledged word. \textbf{REWRITE}

With all the demarcation criteria taken into cosideration, the category of synthetic diminutives does not seem to form a clear case: while there is a noticeable overall tendency towards derivation, diminutives seem to exhibit properties from both domains. This is observed for Italian by \citeauthor{Scalise+1986} (\citeyear{Scalise+1986}), who postulates an intermediate morphological category of what he calls \textit{evaluative suffixes} based on the evidence from diminutive suffixes such as \textit{-ino}. He counts diminutive, augmentative, pejorative, and other suffixes in this category and outlines their shared peculiarities as follows:

\begin{enumerate}
\item[a.] Effecting semantic change of the base
\item[b.] Capability for recursive application
\item[c.] Attaching between derivation and inflection
\item[d.] Possibility of recursive attachment to the same base
\item[e.] Not effecting word class changes of the base
\item[f.] Not effecting the subcategorisation features of the base
\end{enumerate}



\section{Diminutives in Dutch}
\label{sec:dutchdimsuffix}
\subsection{Diminutives in Germanic Languages}
\begin{itemize}
\item Formal patterns: affixation and periphrasis (\cite{Alexiadou+Lohndal}, \cite{Schneider+2003})
\item Quantificational aspects of meaning between German, Dutch, and Afrikaans (\cite{DeBelder+2011a}) 
\end{itemize}
\subsection{The Dutch Diminutive Suffix}
\textbf{-tje and -ke: dialectal variation, patterns of meaning (\cite{VanderHulst+2008}, \cite{DeBelder+2022}), discuss the different semantic interpretations}
\subsection{Between Inflection and Derivation}
\textbf{the alignment of the Dutch diminutive along the inflection/derivation criteria, \cite{Booij+2000})}