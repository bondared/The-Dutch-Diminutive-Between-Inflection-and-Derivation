\chapter{Background}\label{background} 
In order to lay the groundwork for this thesis, we need to establish the parameters typical for diminutive wordforms across languages and use those to arrive at a prototypical diminutive wordform. We will start with a working definition adapted from \cite{Schneider+2003} and use it to examine the formal and semantic properties associated with diminutives. 

\section{The Diminutive}

At the start of his seminal \citeyear{Schneider+2003} work on English diminutives, \citeauthor{Schneider+2003} provides the traditional applied definition of diminutives as "words which denote smallness and possibly also expressing an attitude" \parencite[p.4]{Schneider+2003} and outlines the criteria used to describe a diminutive form, distinguishing between the formal and the semantic ones. 

\subsection{Formal Criteria}
\label{subsec:2-formal-crit}
The formal criteria describe the variables involved in the process of diminutive formation, broadly conceptualised as a function that takes a word and performs an operation on it in order to produce a diminutive formally and/or semantically related to the initial word. The variables involved in this function, then, are the input and output word, as well as the type of unilateral morphological process that transforms one into the other. Of particular theoretical interest are the possible changes in word class between the input and the output word induced by the morphological process. 

There is considerable cross-linguistic variation in terms of what can function as an input to the diminutive function. Nouns are intuitively the go-to category when one thinks of diminutive formation: words such as the English \textit{dog}-\textit{doggie} "dog+DIM", or the German \textit{Katze}-\textit{Kätzchen} "cat+DIM" are ubiquitous across languages. Additional possible open-class categories include adjectives (cf. Russian \textit{старенький} "old+DIM", from \textit{старый} "old", or Italian \textit{azzurrino} "blue+DIM, pale blue", from \textit{azzurro} "sky blue"), adverbs (cf. Dutch \textit{netjes} "neatly+DIM", from \textit{net} "neatly"), and even verbs (cf. German \textit{drängeln} "push+DIM, jostle", from \textit{drängen} "press, push", and Italian \textit{fisciettare} "whistle+DIM", from \textit{fischiare} "whistle"). Closed-class categories such as interjections and prepositions are also known to sporadically provide the base of a diminutive form: cf. Russian \textit{приветик} and German \textit{hallöchen} "hello+DIM", respectively from \textit{привет} and \textit{hallo} "hello", as well as Dutch \textit{uitje} "outing" from \textit{uit} "out" and \textit{ommetje} "detour, stroll" from \textit{om} "around". In even rarer cases, whole phrasal structures can be co-opted by the process of diminutive formation, such as in the case of the Dutch \textit{onderonsje} "private conversation", from \textit{onder} "under, among" + \textit{ons} "us". Despite this broad selection of available categories, the noun forms the most frequent one by far and considered to be the prototypical input word class by \cite{Schneider+2003}, who establishes the following hierarchy:

\begin{exe}
\ex \label{ex:inputhierarchy}
Word class input in diminutive formation (adopted from \citeauthor{Schneider+2003} \citeyear{Schneider+2003}:6)
\begin{enumerate}
\item[i.] open-class words > closed-class words
\item[ii.] nouns and adjectives > verbs and adverbs
\item[iii.] nouns > adjectives
\end{enumerate}
\end{exe}

If the prototypical base of a diminutive word is a noun, then what is the prototypical output? Most of the examples so far have exhibited no change in category between input and output. However, there is some evidence of nominalisation such as the Dutch \textit{uitje} and \textit{ommetje}, where prepositional bases are changed into nouns. The same process often applies to deadjectival forms: cf. English \textit{sweet}-\textit{sweetie}, Dutch \textit{groen} "green" - \textit{groentje} "novice, rookie", or Italian \textit{rosso} "red, ginger" - \textit{rossino} "redhead". Based on these observations, it seems that word class is either retained from the base form or shifts to nominal. \citeauthor{Schneider+2003} (\citeyear{Schneider+2003}) observes that the former tends to apply as a general rule, with the latter otherwise being the case. A typical diminutive form is therefore a noun produced from a noun. However, word class changes are common as well, resulting in nominalised forms.

With the input and output parameters having been examined, we can focus on the selection of morphological processes that are responsible for the process of diminutive formation across languages. Again, based on the examples presented so far, it would be safe to assume that affixation (or, more precisely, suffixation) is the sole engine behind this process. However, as the non-exhaustive sample in (\ref{ex:dimformpatterns}) shows, this is far from the whole picture even in the case of a single language like English:

\begin{exe}
\ex \label{ex:dimformpatterns}
A sample of English diminutive formation mechanisms 
\begin{xlist}
    \ex suffixation: \textit{dog} - \textit{doggie}
    \ex prefixation: \textit{pig} - \textit{minipig}
    \ex truncation: \textit{Elizabeth} - \textit{Liz}
    \ex (rhyming) reduplication: \textit{Jeff Banks} - \textit{Jeff Banksy-Wanksy}
    \ex compounding: \textit{boy} - \textit{baby boy}
    \ex analytic formation: \textit{lady} - \textit{little lady}
    \ex any combination thereof\footnote{For truly outstanding examples of this in action, the reader is kindly referred to the \href{https://abitoffryandlaurie.co.uk/sketches/pooch}{transcript of the "Pooch" sketch} featured on the UK TV show \textit{A bit of Fry and Laurie}.}
\end{xlist}
\end{exe}

The formation mechanisms in (\ref{ex:dimformpatterns}) fall into two overarching categories: those that directly modify the root or stem of the base word, and those that combine another word like \textit{little} with the stem to seemingly form a whole phrase. The function of \textit{little} and similar terms as a tool of diminutive formation in English is reflective of a common trend across languages. The categorical split is observed in \citeauthor{Schneider+2003} (\citeyear{Schneider+2003}), who uses the term "synthetic diminutive formation" for the former and "analytic diminutive formation" for the latter broad category. 

\citeauthor{Schneider+2003} makes an additional distinction between derivational and inflectional affixation: while purely derivational affixes do not effect a change in grammatical information such as noun class or gender, those that do are considered to belong in the domain of inflection. While English lacks gender, its sister language German does not, and can provide a good example. All German nouns have an associated gender value, such as \textit{Löffel} "spoon" being coded masculine and \textit{Gabel} "fork" feminine. However, in the process of diminutive formation involving either of the suffixes \textit{-chen} and \textit{-lein}, a mandatory gender switch takes place: \textit{Löffellein} "spoon+DIM" and \textit{Gäbelchen} "fork+DIM" are both neuter, constituting a case of inflectional suffixation. \citeauthor{Schneider+2003} goes on to claim in passing that the distinction between inflectional and derivational affixation routes in diminutive formation applies parametrically for every language, implying that diminutive affixes are all either exclusively derivational or inflectional in any given language, but notes that this claim requires more evidence, preferably from non-Indo-European languages.

Both distinctions will be touched upon again in the subsequent sections, with the inflectional-derivational distinction especially discussed in Subsection~\ref{subsec:2-infl-deriv} and section~\ref{sec:3-split-dim}. For now, it suffices to establish derivational suffixation as the most frequent mechanism of diminutive formation across languages (\citeauthor{Schneider+2003} \citeyear{Schneider+2003}). Formally, then, a prototypical diminutive is a noun formed from a noun by derivational suffixation. With the formal parameters now outlined and their default values established, let us consider the patterns of meaning typically associated with diminutives next.

\subsection{Semantic Criteria}
\label{subsec:2-semantic-crit}
The linguistic term "diminutive" intuitively has something to do with size comparison, and while precise definitions of the whole category may vary depending on the theoretical framework, the scalar aspect of diminutive meaning is invariably the central one. This is captured in the traditional applied definition provided by \citeauthor{Schneider+2003}: diminutives have long been thought of as "words which denote smallness and possibly also expressing an attitude" (\citeauthor{Schneider+2003} \citeyear{Schneider+2003}:4), with the attitudinal component expressing either a positive or a negative value. From this definition alone, one can visualise each of the two components as expressing a scalar relationship: a \textit{doggie} is a small \textit{dog}, while \textit{Lizzie} is a term of endearment for someone called \textit{Elizabeth}, indicating a positive attitude of the speaker towards them. The notion of scalarity being central to diminutive meaning warrants a deeper examination below.

As far as the size component is concerned, the semantic relationship between a diminutive and its base word is straightforwardly that of hyponymy: a \textit{doggie}, being a small \textit{dog}, is part of the category \textsc{dog}. Therefore, diminutives tend to express relative smallness as defined by the semantic category they belong to. For animate categories, this fact carries an additional consequence of diminutives typically denoting non-adult members (cf. \textit{duck}-\textit{duckling}, \textit{prince}-\textit{princeling}), with the latter example additionally implying immaturity or irrelevance. In purely semantic terms, the diminutive denotes something in some way inferior by comparison, implying a stable relationship between its denotation and the category member it is being compared to.

However, this observation does not mean that a speaker invariably refers to a small duck when saying the word \textit{duckling}, or even perceives the entity they are referring to as small: according to \citeauthor{Schneider+2003} (\citeyear{Schneider+2003}), the speaker's intent plays a crucial role in the total meaning of a diminutive form. In other words, the choice to refer to something as small or inferior is sometimes more relevant than its actual relative size. The example of \textit{princeling} above serves as a good showcase of this, usually being meant derogatively. Therefore, it seems that even the most prototypical aspect of diminutive meaning is somewhat modulated by pragmatic considerations.

The attitude component, established in the above definition as optional, describes the speaker's feelings about the referent. This part of diminutive meaning, too, is inherently scalar: the referent is regarded either more or less favourably by the speaker than some default value. It stands to reason, then, that this component would also be subject to pragmatic effects. Indeed, this is often the case to an even greater degree, with the amount of (dis-)favour sometimes being entirely dependent on the context. Consider the example in (\ref{ex:kereltje}) using the Dutch diminutive suffix \textit{-tje}: \textit{kereltje} "lad+DIM" can be used either as a term of endearment or an expression of dismissiveness.

\begin{exe}
\ex \label{ex:kereltje} 
\begin{xlist}
\ex \gll
\textit{Hij} \textit{is} \textit{een} \textit{raar} \textit{kereltje}.\\
He	is	a  strange	lad-DIM\\
\trans "He is a strange little lad" (dismissively)

\ex \gll
\textit{Hij} \textit{is} \textit{een} \textit{lief} \textit{kereltje}.\\
He	is	a  sweet	lad-DIM\\
\trans "He is a sweet little lad" (endearingly)
\end{xlist}
\end{exe}

The same holds true of \textit{little lad} in the English gloss. Here, \textit{little} seems to primarily express an attitudinal component, and using the purely size-denoting \textit{small} to express a lexical meaning similar to that of \textit{little} does not yield the same interpretation, as shown in (\ref{ex:littlelad}):

\begin{exe}
\ex \label{ex:littlelad}
\begin{xlist}
\ex \textit{He is a strange little lad}. (dismissively)
\ex *\textit{He is a strange small lad}. (dismissively)
\end{xlist}
\end{exe}

Despite the previous examples, attitude is not purely relegated to pragmatics, and some diminutive forms even code it as their primary semantic component. For example, the German word \textit{Väterchen} "father+DIM", from \textit{Vater} "father", acts as a term of endearment regardless of context. The same applies to the word \textit{Mütterlein} "mother+DIM", similarly formed from \textit{Mutter} "mother" via the suffix \textit{-lein}: both mean something like "dear father" and "dear mother", respectively. If anything, "dear X" is the only available meaning for either of the two, as glossing them with "small X" is unlicensed  (\citeauthor{Schneider+2003} \citeyear{Schneider+2003}).

The observations above paint a complex picture of the two primary diminutive meaning components: while both are inherently scalar, one seems to be primarily semantic in nature and the other is mostly governed by pragmatics. However, neither of the two allows for a clear-cut categorisation and often interplays with the other, leaving the prototypical meaning of a diminutive harder to pinpoint. \citeauthor{Schneider+2003} (\citeyear{Schneider+2003}) maintains that smallness is the prototypical component of diminutive meaning but simultaneously concedes that diminutive forms usually express something at "the interface between concepts of quantification and qualification, [combining] aspects of size and attitude" (\citeauthor{Schneider+2003} \citeyear{Schneider+2003}: 4).

An additional aspect of diminutive meaning that logically follows from their quantificational nature is their ability to induce shifts from mass to count nouns in some languages. \citeauthor{Wiltschko+2006} (\citeyear{Wiltschko+2006}) presents evidence from Dutch and German and argues that both languages are able to transform mass nouns like \textit{brood} / \textit{Brot} "bread" or even \textit{slaap} / \textit{Schlaf} "sleep" into their countable counterparts through diminutive formation. Consider the examples from German in (\ref{ex:mass-to-count-to-kind-to-item}): the word \textit{Brot} is a mass noun that acquires a kind reading whenever it is quantified by a process such as pluralisation, but this is as far as the shift goes. In order to talk about an individual roll of bread, the speaker has to form the diminutive \textit{Brötchen}. By contrast, other mass nouns like \textit{Schlaf} do not even have a kind reading, and depend fully on the diminutive for countability.

\begin{exe}
\ex \label{ex:mass-to-count-to-kind-to-item} 
Diminutive quantification in German (adopted from \citeauthor{Wiltschko+2006} \citeyear{Wiltschko+2006})
\begin{multicols}{2}
\begin{xlist}
\ex \gll
\textit{viel} \textit{Brot}\\
much bread\\
\trans "much bread"
\ex \gll
\textit{viel-e} \textit{Brot-e}\\
many-PL bread-PL\\
\trans "many kinds of bread"
\ex \gll
\textit{viel-e} \textit{Bröt-chen}\\
many-PL bread-DIM.PL\\
\trans "many bread rolls/sandwich"
\columnbreak
\ex \gll
\textit{viel} \textit{Schlaf}\\
much sleep\\
\trans "much sleep"
\ex \gll
*\textit{viel-e} \textit{Schläf-e}\\
many-PL sleep-PL\\
\trans ~~~~~~
\ex \gll
\textit{viel-e} \textit{Schläf-chen}\\
many-PL sleep-DIM.PL\\
\trans "many sleeps/naps"
\end{xlist}
\end{multicols}
\end{exe}

\citeauthor{Wiltschko+2006} thus ascribes an individuating function to the German and Dutch diminutive suffixes and notes that this effect is observed in other languages such as Yiddish, Ojibwa, Ewe, Baule, and Cantonese. She then compares the German diminutive suffix to numeral classifiers such as \textit{Stück} "piece" in \textit{12 Stück Brot} "12 pieces of bread" and uses the evidence of the shared pattern to argue that the suffix formally functions as a classifier as well. This has implications for the morphological structure of such diminutives, with the diminutive suffix proposed to act as the head of its phrase.

All the patterns of diminutive meaning discussed so far have been compositional in their nature: no matter the morphological process or degree of modification to the original meaning, it was still fairly transparently available. However, sometimes a formally diminutive word does not seem to have an independently-occurring base, e.g. German \textit{Kaninchen} "rabbit" or Dutch \textit{doetje} "softie". Words like this are called \textit{diminutiva tantum} and exhibit a wide range of meanings, rarely retaining the typical diminutive meaning components. In other cases the base word does exist, yet there is no observable relation between the denotations of the base and its diminutive anymore, e.g. Italian \textit{casino} "brothel", from \textit{casa} "house". A further complication is that words seemingly formed with the same suffix can span the range of fully transparent, as in (\ref{ex:casino1}) to fully opaque in (\ref{ex:casino4}).\footnote{Both a cell phone and a bread roll are arguably smaller, more compact versions of a telephone and a bread loaf, respectively; however, both \textit{telefonino} and \textit{panino} denote objects with other characteristics beyond their relative size, making their meaning partially non-compositional.} This pattern is attested in a wide variety of languages, from Dutch and Spanish to Polish and Tunisian Arabic (see \citeauthor{DeBelder+etal+2014} \citeyear{DeBelder+etal+2014} for further examples).
%\footnote{Note that there exists a purely compositional noun \textit{casina} "small house" that even employs the same suffix \textit{-in-}. The difference here is the induced change in noun gender; as discussed above, this would indicate that the suffix \textit{-in-} is an inflectional one. However, this quickly calls into question the parametric affixation claim from \citeauthor{Schneider+2003} (\citeyear{Schneider+2003}): either all Italian diminutive suffixes are }. The Italian diminutive suffix \textit{-in-} 
\begin{exe}
\ex \label{ex:casino} 
\begin{multicols}{2}
\begin{xlist}
\ex \label{ex:casino1} \gll
\textit{gatt-ino} \\
cat-DIM \\
\trans "small cat", "kitten"
\ex \label{ex:casino2} \gll
\textit{telefon-ino} \\
phone-DIM \\
\trans "cell phone"
\columnbreak
\ex \label{ex:casino3} \gll
\textit{pan-ino} \\
bread-DIM \\
\trans "bread roll, sandwich"
\ex \label{ex:casino4} \gll
\textit{cas-ino} \\
house-DIM \\
\trans "brothel"
\end{xlist}
\end{multicols}
\end{exe}

The diminutive category, therefore, constitutes a broad semantic range, and the meaning of a diminutive is not always clearly discernible from its formal characteristics or the meanings of its constituent parts alone. Additionally, pragmatic considerations come into play even for some of the most straightforward cases of diminution. However, within the domain of semantic compositionality, a clear pattern of scalarity emerges, with the size and attitude components sharing the dominant position (\citeauthor{Schneider+2003} \citeyear{Schneider+2003}) among other possible aspects of meaning.

\subsection{Inflection and Derivation}
\label{subsec:2-infl-deriv}
One crucial categorical distinction that has been touched upon during the discussion of formal criteria in Subsection~\ref{subsec:2-formal-crit} is that between inflectional and derivational diminutives. Recall that within the domain of synthetic diminutive formation, \citeauthor{Schneider+2003} (\citeyear{Schneider+2003}) sets derivational affixation as the prototypical morphological process and goes to suggest that inflectional and derivational affixation are mutually exclusive: in any chosen language, only one of the two processes is responsible for all of diminutive affixation. While the precise strength of the claim is debatable, as pointed out by \citeauthor{Schneider+2003} himself, the claim points to an important question: what makes a diminutive inflectional or derivational?

Traditionally, derivation and inflection have been thought of as distinct morphological processes, each one responsible for a particular linguistic function. According to \citeauthor{Booij+2000} (\citeyear{Booij+2000}), this functional difference is the crucial one for distinguishing between the two: derivation is employed to form new lexemes, i.e. word entries in the mental lexicon, while inflection is said to take existing lexemes and merely modify their form. Already an assumption becomes clear: a lexicon, i.e. a list of morphological primitives such as free and bound morphemes and possibly even morphologically complex forms with a set lexical meaning, is assumed to form a distinct part of the linguistic system. Derivation, then, is a process that operates on the lexicon and internally modifies it, producing new lexical entries; inflection takes lexical entries and modifies them, but does not expand the lexicon.

The fact that both processes access the lexicon, together with them often sharing the same formal means of expression (\citeauthor{Booij+2000} \citeyear{Booij+2000}), somewhat blurs the line between the two; yet they differ in regards to their typical behaviour along a few parameters. While listing and discussing all of those is beyond the purview of this thesis, the reader is referred to \citeauthor{Booij+2000} (\citeyear{Booij+2000}) for a concise overview and \citeauthor{Scalise+1986} (\citeyear{Scalise+1986}) for an in-depth theoretical discussion. What follows is a closer look at the parameters most relevant to diminutive formation.

 As previously observed, the process of diminutive formation can effect a change in word class between the base word and its diminutive form. Word class change also happens to be one of the major parameters between derivation and inflection, with inflectional processes typically seen as operating within syntactic categories while derivational ones optionally induce the change. Note that there are counterexamples when it comes to inflection, e.g. verb forms inflected for infinitive functioning as nouns, as well as participles being used as attributive adjectives (\citeauthor{Booij+2000} \citeyear{Booij+2000}). Nevertheless, the general pattern is that word class change is optional in cases of derivation and nearly unattested in cases of inflection.

 More extreme changes in meaning have also been traditionally associated with the domain of derivation; this is closely related to the notion of semantic transparency, with inflected forms regarded as comparatively more transparent (\citeauthor{Booij+2000} \citeyear{Booij+2000}). Subsection~\ref{subsec:2-semantic-crit} has established that diminutives as a category can show a broad range of meanings, from transparent to opaque; nevertheless, the prototypical meaning of "small X" is reasonably transparent. With regard to the individuating capabilities of some members, sketched out in \citeauthor{Wiltschko+2006} (\citeyear{Wiltschko+2006}), multiple analysis apply: \citeauthor{Scalise+1986} (\citeyear{Scalise+1986}) states that changes in countability, as well as other semantic features such as animacy, are induced by processes of derivation, while the analysis in \citeauthor{DeBelder+2011a} (\citeyear{DeBelder+2011a}) firmly places individuating diminutives within the realm of inflection, at least for the Germanic languages Afrikaans, Dutch, and German. 

Another criterion for demarcation is the capability for recursive application. Derivational processes are said to have greater recursive capabilities (\cite{Booij+2000}, among others). This follows from the observation that derivation is usually concatenative, with suffixes being able to attach onto one another and form new words step-by-step (such as in \textit{dis-establish-ment-ari-an-ism}), while inflectional morphemes are often fusional, expressing multiple aspects of meaning in one surface form (e.g. \textit{-s} in \textit{buys} coding tense, person, and number all at once). Cases of recursively applying inflection do exist, however, and are very common in typologically agglutinative languages (e.g. the Finnish form \textit{taloissa} "in houses" inflecting the stem \textit{talo} "house" for plural with \textit{-i-} before attaching the inessive suffix \textit{-ssa}). 

Lastly, for multimorphemic words such as diminutives, the relative order of placement has been used as an additional diagnostic to determine the category of a morpheme. This comes from the observation that inflectional morphemes are generally more peripheral to derivational morphemes, attaching only once the process of derivation is complete. Conversely, attaching an inflectional morpheme before the derivational one is typically ungrammatical, e.g. \textit{fish-er-s} is perfectly licensed, but *\textit{fish-s-er} is not. This observation is robust enough to establish a hierarchy of attachment, where derivation always precedes inflection (\cite{Booij+2000}). A quick look at pluralised diminutive forms like the English \textit{dogg-ie-s} and Dutch \textit{kat-jes} "cats+DIM" puts the diminutive suffixes squarely before plural markings. On the other hand, pluralised agentive forms like the Italian \textit{cant-ant-in-i} "singers+DIM", from \textit{cantare} "to sing", and Dutch \textit{onderwijz-er-tje-s} "teachers+DIM", from \textit{onderwijzen} "to teach", show that diminutive suffixes are not on the top of the hierarchy. When present in a word with both a prototypical derivational morpheme, such as an agentive suffix, and a prototypically inflectional plural marking, diminutive morphemes are forced to settle somewhere in-between.

None of the demarcation criteria outlined above is enough to sway the decision between inflection and derivation on its own; what matters is the cumulative strength of all parameters making a convincing pull in a certain direction. Unfortunately for the category of synthetic diminutives, it does not seem to form a clear-cut case: while there is a noticeable overall tendency towards derivation, diminutives seem to exhibit properties from both domains. This is best observed by \citeauthor{Scalise+1986} (\citeyear{Scalise+1986}) in Italian for an entire morphological group he calls \textit{evaluative suffixes}. This group includes diminutive, augmentative, pejorative, and other suffixes, all of which are attested to share the following features based on the examples from the suffix \textit{-ino}:

\begin{exe}
\ex \label{ex:scaliseevals}
Features of evaluative suffixes (ES) as per \citeauthor{Scalise+1986} (\citeyear{Scalise+1986}:132f.)
\begin{xlist}
\ex ES change the meaning of the base word \par
(\textit{gatto} "cat" --- \textit{gattino} "cat+DIM")
\ex ES are capable of applying recursively \par
(\textit{pane} "bread" --- \textit{pan-ino} "sandwich" --- \textit{pan-in-etto} "sandwich+DIM")
\ex ES attach between derivational and inflectional suffixes \par
(\textit{cant-ant-in-i}, sing-er-DIM-PL, "singers+DIM")
\ex A single ES can recursively attach onto itself to a moderate extent \par
(\textit{caro} "dear" --- \textit{car-ino} "dear+DIM" --- \textit{car-in-ino} "dear+DIM+DIM")
\ex ES do not change the word class of the base word \par
(\textit{azurro} "blue" --- \textit{azzurrino} "blue+DIM"; \textit{gatto} "cat" --- \textit{gattino} "cat+DIM")
\ex ES do not change the syntactic properties of the base word \par
(\textit{fischiare} "to whistle", [+trans] --- \textit{fisciettare} "to whistle+DIM", [+trans])
\end{xlist}
\end{exe}

Within the selection of features in (\ref{ex:scaliseevals}), \citeauthor{Scalise+1986} (\citeyear{Scalise+1986}) establishes (a) and (b) to be typical of derivational suffixes and (e) and (f) to be typical of inflectional suffixes. The features (c) and (d) in-between are postulated to be unique for the class of evaluative suffixes alone, and their relative placement in the list reflects \citeauthor{Scalise+1986}'s theoretical proposal for an additional intermediate morphological category of \textit{Evaluative Rules}, nested firmly in-between \textit{Derivational} and \textit{Inflectional Rules}. While this proposal has met its share of criticism since (see \cite{Schneider+2003}:34f. for a summary), evaluative morphology as a category has spread through the linguistic community enough to warrant mention and discussion in general introductory texts like \citeauthor{Booij+2000} (\citeyear{Booij+2000}) and \citeauthor{Schneider+2003} (\citeyear{Schneider+2003}).

All in all, diminutives form a complex group of wordforms that broadly vary in terms of their formal and semantic criteria, leaving a lot of space for discussion around the precise definition of the term (see \cite{Schneider+2003} for an outline of contemporary approaches). In addition, the group's relationship with derivation and inflection and its place within the hierarchy of morphological mechanisms is at best unclear, prompting some approaches to adopt an extra morphological category in order to account for the group's idiosyncrasies. Nevertheless, scholars like \citeauthor{Schneider+2003} postulate a clear trend based on observed relative frequency: a diminutive form is more often than not a noun formed from a noun via the process of derivational affixation and expressing smallness and/or attitude based on the speaker's intent. What follows is an investigation of Dutch diminutive morphology with a particular focus on the suffix \textit{-tje} as the most frequent and typical tool of diminutive formation.
 
\section{Diminutives in Dutch}
\label{sec:dutchdimsuffix}
Dutch is a Germanic language closely related to Afrikaans, English, and German. As relatives, these languages share the same morphological patterns of diminutive formation, broadly categorised as synthetic and analytic. The former relies on suffixes while the latter employs a kind of compounding process involving forms like \textit{little} or \textit{baby}, etc. for meaning modification. As \citeauthor{Alexiadou+Lohndal} (\citeyear{Alexiadou+Lohndal}) attest, both types are typically found in most Germanic languages, with the Mainland Scandinavian subgroup forming the only group differing from the rest of Germanic. \citeauthor{Alexiadou+Lohndal} argue on the basis of data from Norwegian that the suffixal determiners typical for Mainland Scandinavian languages prevent those from productively using diminutive suffixation, instead making them solely reliant on analytic processes of compounding. Dutch, by comparison, is able to productively employ both mechanisms, albeit with different degrees of productivity. The current analysis focuses on the more productive synthetic mechanism of diminutive suffixation.

\subsection{The Dutch Diminutive Suffix}
\citeauthor{Schneider+2003} (\citeyear{Schneider+2003}) counts fourteen diminutive suffixes for English. Despite the wide selection of suffixes, English is nonetheless considered to have relatively unproductive diminutive morphology (\citeauthor{Alexiadou+Lohndal}, \citeyear{Alexiadou+Lohndal}). In stark contrast to English, Dutch patterns with German, as the productive capabilities of diminutive suffixes in both languages are well-attested. 

Just like German employs \textit{-chen}, \textit{-lein}, and \textit{-ling} (\citeauthor{Alexiadou+Lohndal}, \citeyear{Alexiadou+Lohndal}), Dutch, too, offers a selection of diminutive suffixes; however, unlike in German, the choice of suffix largely depends on the speaker's region of origin. While \textit{-tje}, together with its allomorphs, is regarded as the standard suffix, Northern Dutch additionally employs the informal suffix \textit{-ie} (e.g. \textit{boek} "book" --- \textit{boekie} "book+DIM", \textit{gek} "crazy" --- \textit{gekkie} "lunatic") and Southern Dutch, particularly the variations spoken in Belgium, uses and even prefers the suffix \textit{-ke} and its allomorphs (\cite{Vandekerckhove+2005}). While this thesis focuses on the Standard Dutch diminutive suffix \textit{-tje}, all three largely share the same formal and semantic features, with the main difference between them being regiolectal.

It is important to note that all examples of the suffix \textit{-tje} so far have used the italicised form as an abstract stand-in for the Standard Dutch suffix, assuming \textit{-tje} as the underlying morpheme with phonologically conditioned surface allomorphs \textit{-je}, \textit{-pje}, \textit{-kje} and \textit{-etje}.\footnote{While it technically would be more correct to spell the allomorphs out as [-jə], [-pjə], [-kjə], and [-ətjə], respectively, Dutch orthography is just as capable of reflecting the sound changes in this particular suffix as the phonetic alphabet, and will therefore be used as shorthand in cases like this.} There is some debate among researchers as to what should be assumed as the underlying form; however, this discussion lies beyond the scope of the current work. The working assumption follows \citeauthor{VanderHulst+2008} (\citeyear{VanderHulst+2008}), who proposes the following generalisations regarding the hierarchy of diminutive allomorph selection:

\begin{exe}
\ex \label{ex:tjeallo}
Generalisations on allomorphy of \textit{-tje} (\citeauthor{VanderHulst+2008}, \citeyear{VanderHulst+2008}:1291)
\begin{xlist}
\ex after short stressed vowels followed by a sonorant consonant: [-ətjə] \par
(\textit{bal} "ball" --- \textit{balletje}, \textit{kam} "comb" --- \textit{kammetje}, \textit{kar} "cart" --- \textit{karretje})
\ex after final obstruent: [-jə] \par
(\textit{baas} "boss" --- \textit{baasje}, \textit{kat} "cat" --- \textit{katje}, \textit{vriend} "friend" --- \textit{vriendje})
\ex elsewhere [-pjə] after /m/, [-kjə] after /ŋ/, [-tjə] in all other cases \par
(\textit{arm} "arm" --- \textit{armpje}, \textit{koning} "king" --- \textit{koninkje}, \textit{leeuw} "lion" --- \textit{leeuwtje})
\end{xlist}
\end{exe}

The Dutch diminutive suffixes adhere to the typical formal patterns estalished in Subsection~\ref{subsec:2-formal-crit}. The noun most typically serves as both the input and output category; however, as the examples in in (\ref{ex:tjekes}) from \citeauthor{DeBelder+2022} (\citeyear{DeBelder+2022}) show, the diminutive suffixes are capable of attaching to and modifying a wide selection of word categories. Note how word class remains unchanged in the examples in (\ref{ex:tjekes}); nominalisations like \textit{dutje} "nap" (from the verb \textit{dutten} "to nap"), \textit{liefje} "sweetheart" (from the adjective \textit{lief} "sweet, dear"), and \textit{uitje} (from the preposition \textit{uit} "out") complete the picture of overall patterns regarding the word class of input and output.

\begin{exe}
\ex \label{ex:tjekes}
Dutch diminutive suffixes \textit{-tje} and \textit{-ke} (\citeauthor{DeBelder+2022}, \citeyear{DeBelder+2022}:101f.)
\begin{multicols}{2}
\begin{xlist}
\ex \label{ex:tjesnoun} \gll
\textit{een} \textit{appel-tje} \\
an apple-DIM \\
\trans "a small apple"
\ex \label{ex:tjesadj} \gll
\textit{Ze} \textit{is} \textit{ziek-je-s} \\
she is ill-DIM-s \\
\trans "She is (somewhat) ill"
\ex \label{ex:tjesadv} \gll
\textit{Ze} \textit{zong} \textit{zacht-je-s} \\
she sang soft-DIM-s \\
\trans "She sang softly"
\ex \label{ex:tjesinterj} \gll
\textit{Hallo-tje-s!} \\
hello-DIM-s \\
\trans "Hello!" (informal)
\columnbreak
\ex \label{ex:kesnoun} \gll
\textit{een} \textit{appel-ke} \\
an apple-DIM \\
\trans "a small apple"
\ex \label{ex:kesadj} \gll
\textit{Ze} \textit{is} \textit{ziek-s-ke-s} \\
she is ill-DIM-s \\
\trans "She is (somewhat) ill"
\ex \label{ex:kesadv} \gll
\textit{Ze} \textit{zong} \textit{zacht-eke-s} \\
she sang soft-DIM-s \\
\trans "She sang softly"
\ex \label{ex:kesinterj} \gll
\textit{Hallo-ke-s!} \\
hello-DIM-s \\
\trans "Hello!" (informal)
\end{xlist}
\end{multicols}
\end{exe}

The nouns formed or modified through diminutive affixation are always of neuter gender. Dutch has a two-way gender system, coding between common and neuter and formally differentiating between the singular nouns with different definite articles and patterns of attributive adjective agreement. Thus, \textit{man} "man, husband" and \textit{vrouw} "woman, wife" both belong to the common gender and share the definite article \textit{de}, while \textit{huis} "house" is neuter and takes \textit{het} as its definite article. When a siingular neuter noun is preceded by an indefinite article and modified by an attributive adjective, the said adjective does not carry the typical ending \textit{-e}: compare \textit{een mooi huis} "a beautiful house" and \textit{een mooi-e man/vrouw} "a beautiful man/woman". Diminutives invariably follow the neuter pattern: \textit{mannetje} "man+DIM, the male of the species" and \textit{vrouwetje} "woman+DIM, the female of the species" are both neuter despite the common gender of their bases, and a phrase like \textit{wat een mooi mannetje} "what a beautiful male (specimen)" will always have its adjective in neuter. Based on these observations, it is safe to assume that Dutch diminutive suffixes assign neuter to nouns they generate or modify.

Semantically, Dutch diminutives seem to primarily reflect the observation in \citeauthor{Schneider+2003} (\citeyear{Schneider+2003}), with the most frequent meaning pattern being some level of interplay between expressing smallness and speaker attitude. As mentioned in Subsection~\ref{subsec:2-semantic-crit}, this basic meaning pattern is intuitively interpretable as scalar. An analysis of \textit{-ke} and \textit{-tje} in \citeauthor{DeBelder+2022} (\citeyear{DeBelder+2022}) extends this scalar interpretation beyond nominal categories of noun and adjective and applies it to adverbs and interjections as well.

Firstly, \citeauthor{DeBelder+2022} (\citeyear{DeBelder+2022}) argues that diminutive suffixes quantify nouns and therefore make them inherently measurable, allowing for scalar relations to be introduced, and then introduce just one such scalar relation by adding a meaning component that places the noun's denotation on the low end of the size scale. Secondly, she observes that Dutch adjectives only allow diminutive modification in predicative contexts and notes the lack of overt morphological difference between uninflected adjectives and the adverbs derived from those adjectives. Thus the inherently gradable adjectives pattern the same as the similarly gradable deadjectival adverbs like \textit{zachtjes} "soft; quiet"/"softly, quietly", in that the addition of diminutive suffixes simply puts the value of the meaning lower on the preexisting scale. Finally, she notes that only stylistically informal interjections allow diminutive suffixation and argues that the addition of the diminutive suffix serves to place the value of the interjection lower on the scale of formality.\footnote{However, \citeauthor{DeBelder+2022} analyses the forms used in interjections as honorific instead of diminutive suffixes, citing differences in the distribution of the epenthetic consonant \textit{-s-} before \textit{-ke} (cf. (\ref{ex:kesadj}) and (\ref{ex:kesinterj}) above) as indicative of different underlying syntactic structures. She nevertheless maintains that both the diminutive and honorific function are fulfilled by the same sufficiently underspecified lexical items \textit{-tje} and \textit{-ke}.} 

The analysis of Dutch diminutive nouns in \citeauthor{DeBelder+2022} (\citeyear{DeBelder+2022}) takes the quantificational aspect of diminutive meaning to be the central one that enables the scalar interpretations in the first place. This goes hand in hand with the general observation in \citeauthor{Wiltschko+2006} (\citeyear{Wiltschko+2006}) and \citeauthor{taalportaal} (\citeyear{taalportaal}) that Dutch diminutive nouns always denote countable entities. In her earlier work on diminutive quantification, \citeauthor{DeBelder+2008} \citeyear{DeBelder+2008} presents the examples in (\ref{ex:dutchquants}) in order to underscore this: diminutive suffixation is unlicensed for mass and kind readings; conversely, it is obligatory for the typical mass noun to receive an item reading. Taking everything into account, then, semantically transparent Dutch diminutives seem to express all the aforementioned aspects of meaning in one convenient package, this package being the diminutive suffix itself. 

\begin{exe}
\ex \label{ex:dutchquants}
Individuation via diminutives in Dutch (summarised from \citeauthor{DeBelder+2008}, \citeyear{DeBelder+2008})
\begin{xlist}
\ex \label{ex:dutchmass}
Mass reading: "I tasted chocolate" \par
\begin{multicols}{2}
\gll
\textit{Ik} \textit{proefde} \textit{chocolade}.\\
I tasted chocolate \\
\trans
\columnbreak
\gll
\#\textit{Ik} \textit{proefde} \textit{chocola-tje}.\\
~I tasted chocolate-DIM \\
\trans
\end{multicols}
\ex \label{ex:dutchkind} 
Kind reading: "I tasted a certain kind of chocolate" \par
\begin{multicols}{2}
\gll
\textit{Ik} \textit{proefde} \textit{een} \textit{chocolade}.\\
I tasted a chocolate \\
\trans
\columnbreak
\gll
\#\textit{Ik} \textit{proefde} \textit{een} \textit{chocola-tje}.\\
~I tasted a chocolate-DIM \\
\trans
\end{multicols}
\ex \label{ex:dutchitem}
Item reading: "I tasted a piece of chocolate" \par
\begin{multicols}{2}
\gll
\#\textit{Ik} \textit{proefde} \textit{een} \textit{chocolade}.\\
~I tasted a chocolate \\
\trans
\columnbreak
\gll
\textit{Ik} \textit{proefde} \textit{een} \textit{chocola-tje}.\\
I tasted a chocolate-DIM \\
\trans
\end{multicols}
\end{xlist}
\end{exe}

When it comes to semantic opacity and non-compositional aspects of meaning, however, Dutch diminutives just as readily present a vast spectrum of idiosyncratic patterns, particularly when the process of diminutive formation involves a change in word class. One frequent pattern of non-compositional meaning is a sort of metonymic shift in nominalised adjectives denoting people or objects with a salient characteristic intially denoted by the adjective in question, e.g. \textit{blond} "blond" --- \textit{blondje} "blond woman", \textit{zoet} "sweet" --- \textit{zoetje} "sweetener", or \textit{cursief} "italic" --- \textit{cursiefje} "small newspaper column" (\cite{taalportaal}). Diminutives formed from other word classes are less frequent and tend to be even more semantically opaque, e.g. the deverbal noun \textit{moetje} "shotgun marriage" from \textit{moeten} "must", the phrasal \textit{twaalfuurtje} "(wrapped) lunch" from \textit{twaalf uur} "twelve o'clock", or the prepositional \textit{ommetje} "detour, stroll" from \textit{om} "around". 

The other side of noncompositionality is the presence of \textit{diminutiva tantum}, i.e. forms whose basis does not occur independently. Such forms are surprisingly frequent in Dutch, with the most prolific example being that of \textit{meisje} "girl". While the meaning of some such forms might have been compositional in earlier stages of the language, e.g. \textit{sprookje} "fairytale" using the now obsolete \textit{sproke} "story" as its base (\cite{taalportaal}), synchronically they pose a question as to their relation to the more compositional, semantically transparent members of the diminutive category, as well as their position within the lexicon.

\subsection{Between Inflection and Derivation}
A different, but related question concerns the morphological nature of the Dutch diminutive suffix: taking all the above infromation into account, does the suffix belong in the category of inflection, or is it derivational? Perhaps it does not fit either and instead occupies an intermediate category like that of evaluative morphology? Each of the three approaches is considered below; as the discussion will quickly show, Dutch is no exception to the special morphological status of diminutives.

DERIVATIONAL APPROACH

On the other hand, the comment in \citeauthor{Schneider+2003} (\citeyear{Schneider+2003}:7f.) regarding the German suffixes \textit{-chen} and \textit{-lein} being inflectional in nature on the basis of them always assigning the neuter gender to their output forms implies that the same should be the case for the Dutch \textit{-tje}, as it invariably assigns one gender to every noun it forms. Taken together with the postulated parametric choice for diminutive affixes being either derivational or inflectional in a given language, this would place \textit{-tje} firmly within the category of inflection. However, this assumption does not take into account all the other word classes the suffix can attach to without forcing a category change. In addition, the varying degrees of semantic opaqueness usually associated with derivational morphology become harder to reconcile with this analysis. 

An argument for the inflectional analysis can be provided on the basis of \citeauthor{Booij+2000} (\citeyear{Booij+2000}) differentiating between inherent and contextual inflection. While contextual inflection is entirely conditioned by the needs of the syntactic structure, e.g. in instances of agreement, inherent inflection is said to be independent of the syntactic context and subject to free choice by the speaker instead. Nominal case agreement with the argument structure of verbs and prepositions is classified as an instance of contextual inflection, while inherent inflection includes comparative and superlative adjective gradation or the selection of tense and aspect of the finite verb. Being less dependent on the syntactic context, inherent inflection is observed by \citeauthor{Booij+2000} (\citeyear{Booij+2000}) to be more idiosyncratic than contextual inflection, exhibiting more opaque patterns of meaning.

As such, the Dutch diminutive could be interpreted as inherently inflectional, coding gender irrespective of the immediate syntactic environment and accommodating less transparent meanings. In addition, \citeauthor{Booij+2000} (\citeyear{Booij+2000}) places inherent inflection between derivation and contextual inflection within the hierarchy of attachment, somewhat satisfying the intuitions about the suffix being somewhere in between the two categories.

Another possible framework one could use to categorise the Dutch diminutive is the evaluative morphology approach from \citeauthor{Scalise+1986} (\citeyear{Scalise+1986}). However, it becomes apparent very quickly that this approach only partially applies to \textit{-tje}. The typically derivational criterion of inducing changes in meaning is fulfilled by the simplest diminutives such as \textit{appeltje} "apple+DIM". In addition, the intermediary position within the hierarchy of attachment is evidenced by formations such as \textit{onderwijz-er-tje-s} "teachers+DIM": the agentive suffix \textit{-er} has to apply before the diminutive, since the form *\textit{onderwijz-etje-r-s} is ill-formed, and the diminutive has to attach before the inflectional plural marker \textit{-s}, since *\textit{onderwijz-er-s-je} is similarly ill-formed. The final characteristic condition that is met by \textit{-tje} is that it does not change the syntactic properties of its bases: as \citeauthor{taalportaal} (\citeyear{taalportaal}) show, phrases like \textit{middel tegen verkoudheid} "remedy against the cold" and \textit{makelaar in koffie} "broker in coffee" can have their constituents diminutised without them losing their internal argument structure, yielding \textit{middeltje tegen verkoudheid} and \textit{makelaartje in koffie}.

For every characteristic feature of evaluative morphology that applies to the diminutive suffix, there is another that does not. One of the inflectional characteristics is instantly unsatisfied, as \textit{-tje} has been shown to freely induce changes in word class. Furthermore, forms featuring recursive application of diminutive suffixes, e.g. *\textit{appel-ke-tje}, are unlicensed in Dutch. A possible explanation for this could be the fact that \textit{-tje} and \textit{-ke} are usually mutually exclusive within a given dialect; however, *\textit{appel-tje-je} turns out to be just as unlicensed. Even forms that would not have the semantic redundancy of *\textit{appel-tje-je} "apple+DIM+DIM", e.g. ones based on a formal diminutive with opaque semantics, do not appear to be licensed in Dutch: *\textit{groen-tje-je} "novice+DIM", originally from \textit{groen} "green", is ungrammatical despite its completely normal semantic content. The last observation rules out another criterion for evaluative morphology, namely that the same suffix can attach to itself up to a certain extent. This is especially damaging to the working assumption of \textit{-tje} being an evaluative suffix, given that one of the two characteristics postulated by \citeauthor{Scalise+1986} (\citeyear{Scalise+1986}) as unique to the category of evaluative morphology does not apply. The overview of the Dutch diminutive suffix assessed in terms of \citeauthor{Scalise+1986}'s approach is summarised in (\ref{ex:tjescalise}):

\begin{exe}
\ex \label{ex:tjescalise}
\textit{-tje} versus an evaluative suffix as per \citeauthor{Scalise+1986} (\citeyear{Scalise+1986}:132f.)
\begin{xlist}
\ex \label{ex:tjescalisea}
Changes the meaning of the base word: \cmark \par
\textit{appel} "apple" --- \textit{appel-tje} "apple+DIM"
\ex \label{ex:tjescaliseb}
Capable of applying recursively: \xmark
\ex \label{ex:tjescalisec}
Attaches between derivational and inflectional suffixes: \cmark \par
\textit{onderwijz-er-tje-s}: teach-er-DIM-PL, "teachers+DIM"
\ex \label{ex:tjescalised}
Can recursively attach onto itself to a moderate extent: \xmark
\ex \label{ex:tjescalisee}
Does not change the word class of the base word: \xmark
\ex \label{ex:tjescalisef}
Does not change the syntactic properties of the base word: \cmark \par
\textit{makelaar in koffie} --- \textit{makelaar-tje in koffie}, "broker+DIM in coffee"
\end{xlist}
\end{exe}

A possible explanation for recursive application being blocked in Dutch comes from the domain of phonotactics: according to \citeauthor{Booij+1998} (\citeyear{Booij+1998}), Dutch complex words are subject to phonological output constraints. For example, Dutch has two plural suffixes, \textit{-s} and \textit{-en}, which are in complementary distribution; the selection of the suffix is largely determined by which one best satisfies the preference for a plural noun ending in a trochee, e.g. \textit{man} --- \textit{mannen}, but \textit{appel} --- \textit{appels}. The phonological generalisation crucial to explaining the recursive attachment problem is the observation that Dutch disprefers multiple unstressed syllables in a row. However, \citeauthor{taalportaal} (\citeyear{taalportaal}) caution against this assumption and provide ample counterexamples such as \textit{tubetje} "tube+DIM" and \textit{kantinetje} "canteen+DIM": both \textit{tube} and \textit{kantine} share a final unstressed syllable, which does not stop the suffix from productively attaching to form the respective diminutive.

In summary, the Dutch diminutive suffix serves as a perfectly average case of diminutive morphology, with all the good and bad that entails: it fits the mold of the prototypical mechanism of diminutisation and diminution and occupies an uncertain position along the lines of derivation and inflection, serving as a challenging case of interfaces between phonology, morphology, syntax, semantics and pragmatics, with countless theory-specific frameworks of analysis seeking to explain its peculiarities in novel terms. While some of those have been outlined above, a particularly interesting suggestion by \citeauthor{DeBelder+etal+2014} (\citeyear{DeBelder+etal+2014}) will be discussed and operationalised in the following chapter.