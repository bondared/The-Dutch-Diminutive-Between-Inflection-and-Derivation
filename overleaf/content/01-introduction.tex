\chapter{Introduction}\label{introduction}
\setcounter{page}{1}

One of the most fascinating aspects of the human capacity for language is its sheer scope, allowing for incredible breadth and depth of expression. It has allowed our species to exchange and put into practice countless novel ideas, possibly using sentences that have never been uttered before. Scholars in the generativist tradition of linguistics assume a humble, yet robust system of rules and structures behind this breadth of expression, serving as the great facilitator of communication. Morphemes, defined as the smallest units of form and meaning, are assumed to be the building blocks of every structure, stored in a list of sorts and retrieved whenever they are required in order to convey some aspect of meaning or ensure formal and grammatical coherence, forming whole classes based on their shared semantics or functions. 

One such class of morphemes are those responsible for forming words like \textit{doggie} and \textit{duckling}, \textit{Freddy} and \textit{baby boy}, words indicating smallness of size or degrees of endearment collectively termed as \textit{diminutives}. These forms are prolific in many languages and fulfill multiple communicative functions, yet they tend to differ across languages with regards to their particular morphological structure and patterns of meaning (\cite{Schneider+2003}). As such, diminutives form a rather broadly defined category, one that includes members preventing a more restrictive definition. While the formal and semantic characteristics of diminutives are relatively easy to pinpoint, the exact relationship of diminutive morphology with the domains of inflection, derivation, and compounding, has been especially challenging to establish, prompting various theoretical analyses (see \cite{Schneider+2003}, for a detailed overview).

This thesis seeks to adopt one such analysis and test its predictions on a suffix observably established as a sufficiently prototypical representative of diminutive morphology, namely the Dutch suffix \textit{-tje}. As the standard diminutive suffix in Dutch, it is the most prolific in the language and enjoys a great degree of productivity, resulting in great variance in meaning ranging from fully transparent to fully opaque, as seen in (\ref{ex:placeholder})\footnote{Glosses in this thesis follow the Leipzig Glossing Rules (\cite{leipzig+rules})}. These patterns are broadly categorised into two categories indicative of the underlying structural differences by \citeauthor{DeBelder+etal+2014} (\citeyear{DeBelder+etal+2014}), who cite evidence from Italian and Hebrew in order to propose the existence of both the inflectional and the derivational types of diminutive morphology. 

\begin{exe}
\ex \label{ex:placeholder} 
\begin{multicols}{2}
\begin{xlist}
\ex \gll
\textit{appel-tje} \\
apple-DIM \\
\trans "small apple"
\ex \gll
\textit{bier-tje} \\
beer-DIM \\
\trans "(serving of) beer"
\columnbreak
\ex \gll
\textit{mann-etje} \\
man-DIM \\
\trans "small man, male (specimen)"
\ex \gll
\textit{groen-tje} \\
green-DIM \\
\trans "novice"
\end{xlist}
\end{multicols}
\end{exe}

The Dutch suffix \textit{-tje} is then assumed to belong to both types simultaneously, attaching in different locations within the word structure depending on its function. This idea of a functional split is put to the test in a lexical decision experiment, with the clear expectation that the differences in internal complexity will be reflected in processing times by Dutch-speaking participants. The experimental results stand to shed some light on the structure of diminutive morphology employed in Dutch and similarly patterning languages, and contribute to the theoretical debate on the function of diminutives via empirical testing.

This thesis is structured as follows: Chapter~\ref{chp:background} delves into the discussion of diminutives in great detail, outlining the general characteristics of the class and settling on a set of values most prototypical of a diminutive morpheme. The Dutch diminutive suffix \textit{-tje} is subsequently introduced and checked against those prototypical values, with the bulk of the discussion dedicated to its unclear positioning in the derivation/inflection debate. Chapter~\ref{chp:proposal} presents a theoretical account that offers to explain unclear positioning by proposing a functional split between derivational and inflectional capabilities fulfilled by the same suffix and formulates the predictions stemming from this proposal regarding its empirical corroboration. A visual word recognition experiment is proposed as the vehicle of hypothesis testing and its design is carefully outlined in Chapter~\ref{chp:study}. The results of this experiment are reported in a non-evaluative fashion in Chapter~\ref{chp:results}, which feeds into the evaluative discussion of the results and limitations of the study in Chapter~\ref{chp:discussion}. Finally, the goals of this thesis are restated and checked against the results in Chapter~\ref{chp:conclusion}, which provides a verdict and outlines the possibilities of further research.