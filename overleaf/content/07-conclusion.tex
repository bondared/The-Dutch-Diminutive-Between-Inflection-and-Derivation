\chapter{Conclusion}\label{chp:conclusion}
This study sought to contribute to the debate on the morphological nature of diminutive suffixes by adopting a promising theoretical analysis and corroborating it via empirical testing. Taking the Dutch diminutive suffix \textit{-tje} as its focus, it adopted a proposal advanced in \citeauthor{DeBelder+etal+2014} (\citeyear{DeBelder+etal+2014}), tentatively referred as the \textit{functional split proposal}, where the diminutive suffix is argued to function either in a derivational or inflectional capacity depending on the needs of the speaker. 

Semantic compositionality served as the key factor in the argument for underlying structural differences under the theory of Distributed Morphology (\cite{Halle+Marantz+1993}, \citeyear{Halle+Marantz+1994}): the suffix in its derivational function was proposed to merge directly with the root below the category-assigning head, rendering the meaning of the resulting construction comparatively noncompositional as well. Conversely, the very same suffix in its inflectional employ as defined by regular and compositional, primarily scalar and quantificational patterns of meaning, was proposed to occupy a functional projection above the category-assigning head. Diminutives formed with the derivational suffix were then concluded to require an extra lexical projection compared to their inflected counterparts, with the added structural complexity predicted to impact lexical recognition times under the assumption of obligatory decomposition of morphologically complex forms (\cite{Taft+1979}).

In order to test this prediction, a visual word recognition experiment was simulated on the basis of the diminutive subcorpus of the Dutch Lexicon Project 2 (\cite{Brysbaert+etal+2016}), testing the effect of diminutive type on reaction times elicited from Flemish Dutch-speaking participants. The observed effects of diminutive type, as well as the controlling variable of morpheme count, were inconclusive, with neither predictor obtaining significance. Interaction effects between diminutive type and surface frequency proved similarly incoherent with the working theory; however, this pattern is possibly attributable to the experiment not controlling for base or stem frequency in addition to surface frequency measures (\cite{Taft+1979}). Together with the observation that all the fixed effects included in the model explained only six per cent of the total variance, this leads us to conclude that additional factors possibly need to be accounted for in subsequent work seeking to corroborate the working hypothesis through similar methods. 

Due to the limitations of the current design and the inadvertent introduction of methodological artifacts, the working hypothesis cannot be rejected outright, and while the findings within this work are inconclusive, they still give some credence to the analysis adopted here. As the first experiment to operationalise the proposal in \citeauthor{DeBelder+etal+2014} (\citeyear{DeBelder+etal+2014}), it serves as a foot in the door of its empirical testing, and one can only hope that a study that better controls for possible confounding factors, e.g. the base frequency of morphemes, the regiolectal background of its participants, and possibly even meaning subregularities among both the inflectional and derivational groups, can produce some very convincing evidence in favour of the functional split analysis of diminutive morphology in the future.