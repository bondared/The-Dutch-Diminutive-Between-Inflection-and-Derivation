\chapter{Proposal}\label{chp:proposal}

What happens in this chapter? \par

\section{The Split Diminutive Proposal}
\label{sec:3-split-dim}
\begin{itemize}
    \item Introduce the analysis in \cite{DeBelder+etal+2014}, \cite{DeBelder+2011a}: Inflectional VS Derivational Diminutive, link back to diminutive semantics, mention \cite{Wiltschko+2006}
    \item Establish the outline of her analysis for the Dutch diminutive as well
\end{itemize}
\section{Distributed Morphology and the Split Diminutive}
\begin{itemize}
    \item Situate the analysis within the broader context of distributed morphology
    \begin{itemize}
        \item Split lexicon (Syntactic Terminals, Vocabulary, Encyclopedia; \cite{Embick+2015})
        \item FULL DECOMPOSITION as a theoretical assumption supported by empirical findings
    \end{itemize}
    \item Finish by introducing the idea of corroborating with psycholinguistic evidence
\end{itemize}
\section{Predictions for Visual Word Processing}
\begin{itemize}
    \item Expand on the idea of experimental corroboration
    \item Discuss full decomposition (cite \cite{Taft+1979} and \cite{Taft+2004}, strengthen with evidence from \cite{Fruchter+Marantz+2015}, \cite{Stockall+Marantz2006} and \cite{Fruchter+etal+2013})
    \item Establish predictions and hypotheses according to De Belder's analysis (Different points of attachment = different mechanisms at play = possibly different cognitive load = different reaction times/accuracy)
    \item Brush on alternative predictions and implications (lexicalist predictions: if a diminutive is a new lexeme, it should be stored in the lexicon and not decomposed into constituent morphemes)
    \item Segue into the methodology part
\end{itemize}

The project is rooted within the framework of Distributed Morphology and based on the theory of there existing two heads that diminutive morphemes can attach to: Size and Lex. As such, the group of diminutive morphemes is posited to be split along the lines of where each attaches (or is "spelled out" even):

\begin{itemize}
\item Inflectional Diminutive: fully productive and compositional, e.g. X + Dim = "small X"; merges above the category head with the category phrase and realises SizeP; responsible for the kind-to-item reading shift.
\item Derivational Diminutive: lexical gaps and non-compositional meaning, e.g. X + Dim = Y (new denotation); merges below the category head with the root and realizes LexP; possibly phonologically irregular allomorph selection.
\end{itemize}

\citeauthor{DeBelder+etal+2014} further predict that the two types of diminutive should be able to attach one after the other, i.e. the inflectional suffix should be able to attach after the derivational. This preditction is corroborated with evidence from \textbf{LANGUAGES}. For cases where the same surface form applies recursively (e.g. Italian \textit{car-in-ino}), it is proposed that both functions are fulfilled by the same exponent, e.g. \textit{-in-} having a derivational function (corroborated by the meaning gap in \textit{care} "dear" --- \textit{carino} "nice") and an inflectional one (\textit{carino} "nice" --- \textit{carinino} "pretty nice"). This analysis explains the observations for evaluative suffixes in \citeauthor{Scalise+1986} (\citeyear{Scalise+1986}), particularly (b) and (d) from \ref{ex:scaliseevals}, in different structural terms. However, it seems that additional factors can play a role in whether a diminutive suffix can attach recursively: a string like \textit{-tje-je} in *\textit{groen-tje-je} is unlicensed. In a footnote, \citeauthor{DeBelder+etal+2014} (\citeyear{DeBelder+etal+2014}) suggest that this restriction might be due to Dutch dispreferring two unstressed syllables in a row, an observation supported by data in \citeauthor{VanderHulst+2008} (\citeyear{VanderHulst+2008}). 


% LEGACY CODE; MIGHT BE USEFUL
\begin{exe}
\ex \label{ex:debeldersdims}
Split structure \par
\Tree [.DivP [ ] [.Div\1 [.Div\0 ] [.SizeP [ ] [.Size\1 [.Size\0 ] [.nP [ ] [.n\1 [.n\0 ] [.LexP [ ] [.Lex\1 [.Lex\0 ] [.$\surd$ !\qsetw{1cm} ] ] ] ] ] ] ] ] ]
\end{exe}

Here is some text

\newpage
\begin{exe}
\ex \label{ex:dims-infl-deriv}
Split structure
\begin{multicols}{2}
\begin{xlist}
\ex \label{ex:dims-infl}
Inflectional diminutive \par \medskip
\Tree [.DivP [ ] [.Div\1 [.Div\0 ] [.SizeP [ ] [.Size\1 [.Size\0 \textit{-tje} ] [.nP [ ] [.n\1 [.n\0 ] [.LexP [ ] [.Lex\1 [.Lex\0 ] [.$\surd$ \textit{bier} ] ] ] ] ] ] ] ] ]
\columnbreak
\ex \label{ex:dims-deriv}
Derivational diminutive \par \medskip
\Tree [.DivP [ ] [.Div\1 [.Div\0 ] [.SizeP [ ] [.Size\1 [.Size\0 ] [.nP [ ] [.n\1 [.n\0 ] [.LexP [ ] [.Lex\1 [.Lex\0 \textit{-tje} ] [.$\surd$ \textit{groen} ] ] ] ] ] ] ] ] ]
\end{xlist}
\end{multicols}
\end{exe}

(\cite{DeBelder+2022}): Phonological domains are syntactically determined. the affix and the stem need to be in the same cyclic domain in the syntactic structure to be spelled out together and thus to form a determining phonological domain. (cf. bloemetje and bloempje, etc.)
(\cite{DeBelder+2022}): four crucial properties of DM: morphology is syntax, late insertion, roots, cyclic structure-building