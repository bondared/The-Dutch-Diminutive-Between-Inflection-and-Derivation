\chapter{The Morphological Question}\label{chp:proposal}

What happens in this chapter? \par

\section{The Split Diminutive Proposal}
\begin{itemize}
    \item Introduce the analysis in \cite{DeBelder+etal+2014}, \cite{DeBelder+2011a}: Inflectional VS Derivational Diminutive, link back to diminutive semantics, mention \cite{Wiltschko+2006}
    \item Establish the outline of her analysis for the Dutch diminutive as well
\end{itemize}
\section{Distributed Morphology and the Functional Split}
\begin{itemize}
    \item Situate the analysis within the broader context of distributed morphology
    \begin{itemize}
        \item Split lexicon (Syntactic Terminals, Vocabulary, Encyclopedia; \cite{Embick+2015})
        \item FULL DECOMPOSITION as a theoretical assumption supported by empirical findings
    \end{itemize}
    \item Finish by introducing the idea of corroborating with psycholinguistic evidence
\end{itemize}
\section{Predictions for Visual Word Processing}
\begin{itemize}
    \item Expand on the idea of experimental corroboration
    \item Discuss full decomposition (cite \cite{Taft+1979} and \cite{Taft+2004}, strengthen with evidence from \cite{Fruchter+Marantz+2015}, \cite{Stockall+Marantz2006} and \cite{Fruchter+etal+2013})
    \item Establish predictions and hypotheses according to De Belder's analysis (Different points of attachment = different mechanisms at play = possibly different cognitive load = different reaction times/accuracy)
    \item Brush on alternative predictions and implications (lexicalist predictions: if a diminutive is a new lexeme, it should be stored in the lexicon and not decomposed into constituent morphemes)
    \item Segue into the methodology part
\end{itemize}

The project is rooted within the framework of Distributed Morphology and based on the theory of there existing two heads that diminutive morphemes can attach to: Size and Lex. As such, the group of diminutive morphemes is posited to be split along the lines of where each attaches (or is "spelled out" even):

\begin{itemize}
\item Inflectional Diminutive: fully productive and compositional, e.g. X + Dim = "small X"; merges above the category head with the category phrase and realises SizeP; responsible for the kind-to-item reading shift.
\item Derivational Diminutive: lexical gaps and non-compositional meaning, e.g. X + Dim = Y (new denotation); merges below the category head with the root and realizes LexP; possibly phonologically irregular allomorph selection.
\end{itemize}

% LEGACY CODE; MIGHT BE USEFUL
\begin{figure}[htp]
\label{fig:debeldersdims}
\Tree [.DivP [ ] [.Div\1 [.Div\0 ] [.SizeP [ ] [.Size\1 [.Size\0 ] [.nP [ ] [.n\1 [.n\0 ] [.LexP [ ] [.Lex\1 [.Lex\0 ] [.$\surd$ !\qsetw{1cm} ] ] ] ] ] ] ] ] ]
\caption{Proposal for the diminutive structure}
\end{figure}