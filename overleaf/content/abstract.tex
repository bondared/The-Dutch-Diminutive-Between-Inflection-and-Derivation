\begin{abstractDE}
Diminutivmorphologie stellt ein sprachenübergreifend faszinierendes Phänomen dar, das Wissenschaftler jahrzehntelang verwirrte und zu verschiedenen theoretischen Analysen führte (\cite{Schneider+2003}). Innerhalb generativistischer Ansätze besteht ein besonderer Streitpunkt darin, ob Diminutivmorpheme ihrer Natur nach flektierend oder wortbildend sind, wobei Gelehrte wie \citeauthor{Scalise+1986} (\citeyear{Scalise+1986}) sich stattdessen sogar für einen intermediären Vorschlag entscheiden. Eine viel versprechende Analyse von \citeauthor{DeBelder+etal+2014} (\citeyear{DeBelder+etal+2014}) argumentiert dafür, dass Diminutivmorpheme je nach ihrer Funktion in der morphologischen und semantischen Struktur sowohl zur Wortbildung als auch zur Flektion gehören. Diese Analyse wird in der aktuellen Arbeit diskutiert und für empirische Untersuchung operationalisiert. Ein virtuelles lexikalisches Entscheidungsexperiment wird durchgeführt zu den Auswirkungen des niederländischen Diminutivsuffixes \textit{-tje} je nach seiner Funktion als Teil der Flexion oder der Wortbildung. Die experimentellen Ergebnisse zeigen ein verwirrendes Bild, wobei einige Muster möglicherweise als Artefakte des experimentellen Designs zu erklären sind. Als solche sind die Ergebnisse nicht schlüssig, dienen aber als einführender Ausflug in experimentelle Bestätigungsbemühungen des Vorschlags in \citeauthor{DeBelder+etal+2014} (\citeyear{DeBelder+etal+2014}).
\end{abstractDE}

\vfill

\begin{abstractEN}
Diminutive morphology constitutes a fascinating phenomenon across languages, puzzling scholars for decades and prompting various theoretical analyses (\cite{Schneider+2003}). Within generativist approaches, a point of particular contention is whether diminutive morphemes are inflectional or derivational in nature, with scholars like \citeauthor{Scalise+1986} (\citeyear{Scalise+1986}) even opting instead for an intermediary proposal. A promising analysis by \citeauthor{DeBelder+etal+2014} (\citeyear{DeBelder+etal+2014}) instead argues for diminutive morphemes being both derivational and inflectional depending on their function in the morphological and semantic structure. This analysis is discussed in the current work and operationalised for empirical testing involving a virtual lexical decision experiment on the effects of the Dutch diminutive suffix \textit{-tje} as defined by its inflectional or derivational function. The experimental findings present a puzzling picture, with some patterns possibly introduced as artifacts of the experimental design. As such, the results are inconclusive but serve as an introductory foray into experimental corroboration efforts of the proposal in \citeauthor{DeBelder+etal+2014} (\citeyear{DeBelder+etal+2014}).
\end{abstractEN}

\vfill